%% start of file `template.tex'.
%% Copyright 2006-2013 Xavier Danaux (xdanaux@gmail.com).
%
% This work may be distributed and/or modified under the
% conditions of the LaTeX Project Public License version 1.3c,
% available at http://www.latex-project.org/lppl/.


\documentclass[11pt,a4paper,sans]{moderncv}        % possible options include font size ('10pt', '11pt' and '12pt'), paper size ('a4paper', 'letterpaper', 'a5paper', 'legalpaper', 'executivepaper' and 'landscape') and font family ('sans' and 'roman')

% moderncv themes
\moderncvstyle{casual}                             % style options are 'casual' (default), 'classic', 'oldstyle' and 'banking'
\moderncvcolor{blue}                               % color options 'blue' (default), 'orange', 'green', 'red', 'purple', 'grey' and 'black'
%\renewcommand{\familydefault}{\sfdefault}         % to set the default font; use '\sfdefault' for the default sans serif font, '\rmdefault' for the default roman one, or any tex font name
%\nopagenumbers{}                                  % uncomment to suppress automatic page numbering for CVs longer than one page

% character encoding
\usepackage[utf8]{inputenc}                       % if you are not using xelatex ou lualatex, replace by the encoding you are using
%\usepackage{CJKutf8}                              % if you need to use CJK to typeset your resume in Chinese, Japanese or Korean

% adjust the page margins
\usepackage[scale=0.75]{geometry}
%\setlength{\hintscolumnwidth}{3cm}                % if you want to change the width of the column with the dates
%\setlength{\makecvtitlenamewidth}{10cm}           % for the 'classic' style, if you want to force the width allocated to your name and avoid line breaks. be careful though, the length is normally calculated to avoid any overlap with your personal info; use this at your own typographical risks...

% personal data
\name{Federico Agustín}{Caccia}
\title{Currículum Vit\ae, Julio 2018}                               % optional, remove / comment the line if not wanted
%\address{Av. Bustillo 9500, San Carlos de Bariloche}{CP:8400}{Argentina}%% optional, remove / comment the line if not wanted; the "postcode city" and and "country" arguments can be omitted or provided empty
%\phone[mobile]{+54~9~3476~623177}                   % optional, remove / comment the line if not wanted
%\phone[fixed]{+2~(345)~678~901}                    % optional, remove / comment the line if not wanted
%\phone[fax]{+3~(456)~789~012}                      % optional, remove / comment the line if not wanted
%\email{federicoagustincaccia@gmail.com}                               % optional, remove / comment the line if not wanted
%\homepage{www.github.com/fedecaccia}                         % optional, remove / comment the line if not wanted
%\nacinfo{Fecha de nacimiento: 08/02/1989}                 % optional, remove / comment the line if not wanted
%\extrainfo{Estado civil: soltero}                 % optional, remove / comment the line if not wanted
%\dniinfo{DNI: 34297997}                 % optional, remove / comment the line if not wanted
\photo[64pt][0.4pt]{me}                       % optional, remove / comment the line if not wanted; '64pt' is the height the picture must be resized to, 0.4pt is the thickness of the frame around it (put it to 0pt for no frame) and 'picture' is the name of the picture file
%\quote{Ingeniero nuclear}                                 % optional, remove / comment the line if not wanted

% to show numerical labels in the bibliography (default is to show no labels); only useful if you make citations in your resume
%\makeatletter
%\renewcommand*{\bibliographyitemlabel}{\@biblabel{\arabic{enumiv}}}
%\makeatother
%\renewcommand*{\bibliographyitemlabel}{[\arabic{enumiv}]}% CONSIDER REPLACING THE ABOVE BY THIS

% bibliography with mutiple entries
%\usepackage{multibib}
%\newcites{book,misc}{{Books},{Others}}
%----------------------------------------------------------------------------------
%            content
%----------------------------------------------------------------------------------
\begin{document}
%\begin{CJK*}{UTF8}{gbsn}                          % to typeset your resume in Chinese using CJK
%-----       resume       ---------------------------------------------------------
\makecvtitle

\section{Información personal}

\cvitem{}{\textbf{Nombre:} \textit{Federico Agustín Caccia}}

\cvitem{}{\textbf{Fecha y lugar de nacimiento:} \textit{8 de Febrero de 1989, Corrientes, Argentina}}

\cvitem{}{\textbf{DNI:} \textit{34297997}}
%\cvitem{}{\texxtbf{Passport:} \textit{AAE717772}

\cvitem{}{\textbf{Domicilio:} \textit{Av. Bustillo 9500, San Carlos de Bariloche (CP:8400), Argentina}}

\cvitem{}{\textbf{Estado civil:} \textit{Soltero}}

\cvitem{}{\textbf{Número de teléfono:} \texttt{+}\textit{54 9 3476 623177}}

\cvitem{}{\textbf{Email:} \textit{\href{mailto:federicoagustincaccia@gmail.com}{federicoagustincaccia@gmail.com}}}

\cvitem{}{\textbf{Github:} \url{www.github.com/fedecaccia}}

\cvitem{}{\textbf{Linkedin:} \url{www.linkedin.com/in/fedecaccia}}

\cvitem{}{\textbf{ResearchGate:} \url{www.researchgate.net/profile/Federico\_Caccia2}}



\section{Formación académica}
\cventry{2017}{Magíster en Ingeniería}{Instituto Balseiro, Universidad Nacional de Cuyo y Comisión Nacional de Energía Atómica}{San Carlos de Bariloche}{}
{Tesis: \href{http://ricabib.cab.cnea.gov.ar/654/}{\textit{Acoplamiento Multiescala en Cálculos Fluidodinámicos}}.\\
Director: Dr. Enzo A. Dari.}  % arguments 3 to 6 can be left empty

\cventry{2014}{Ingeniero Nuclear}{Instituto Balseiro, Universidad Nacional de Cuyo y Comisión Nacional de Energía Atómica}{San Carlos de Bariloche}{}
{Tesis: \href{http://ricabib.cab.cnea.gov.ar/468/}{\textit{Diseño Conceptual de un Reactor Rápido.}} \newline{}
Director: Dr. Eduardo Villarino.}  % arguments 3 to 6 can be left empty

\cventry{2011}{Estudiante en Ingeniería Civil}{Facultad de Ciencias Exactas, Ingeniería y Agrimensura, Universidad Nacional de Rosario}{Rosario}{}
{Cursados los dos primeros años de carrera hasta obtener la beca de grado en el Instituto Balseiro}

%\cventry{2008}{Estudiante en Psicología}{Facultad de Psicología, Universidad Nacional de Rosario}{Rosario}{}{Cursados los dos primeros años de carrera}

\cventry{2006}{Bachiller Polimodal en Economía y Gestión de las Organizaciones}{Escuela de Enseñanza Media Particular Incorporada nº 8083 \textit{San Carlos}}{San Lorenzo}{}{}

%\section{Master thesis}
%\cvitem{title}{\emph{Title}}
%\cvitem{supervisors}{Supervisors}
%\cvitem{description}{Short thesis abstract}

\cventry{2018}{Investigación en Blockchain}{CoinFabrik}{Buenos Aires, Argentina}{}
{CoinFabrik es una empresa de software enfocada en tecnologías blockchain, FinTech y desarrollo de contratos inteligentes.
Referencias: Ing. Sebastian Raul Wain (\href{mailto:sebastian.wain@nektra.com}{sebastian.wain@nektra.com}).
\begin{itemize}%
\item Proyecto: Análisis cuantitativo de criptoactivos.
	\begin{itemize}%
	\item Duración: Enero 2018 - Julio 2018.
	\item Descripción: Análisis de correlaciones y cointegraciones en criptoactivos. 
	Análisis de estrategias de \textit{trading}, principalmente enfocado en arbitraje estadístico y algoritmos de tipo \textit{mean reversion}.	
	\item Responsabilidades y tareas desarrolladas: análisis de datos y desarrollo de código.
	\end{itemize}
\item Proyecto: Detección \textit{online} de noticias importantes.
	\begin{itemize}%
	\item Duración: Febrero 2018 - Abril 2018.
	\item Descripción: Desarrollo de un código de \textit{incremental clustering} usando técnicas de procesamiento de lenguage natural.
	El programa está compuesto por un \textit{web scrawler} y el principal algoritmo que se encarga de realizar \textit{online clustering} sobre \textit{breaking news,
	tweets} y otros medios sociales.
	\item Responsabilidades y tareas desarrolladas: investigación y desarrollo de código.
	\end{itemize}
\item Proyecto: Mofiler.
	\begin{itemize}%
	\item Duración: Junio 2018 - Julio 2018.
	\item Descripción: Mofiler es una plataforma descentralizada para el comercio masivo de datos generados a patir de millones de dispositivos.	
	\item Responsabilidades y tareas desarrolladas: Economía de \textit{tokens}, que incluye valuación de MOFI \textit{utility tokens} y MOFX \textit{security tokens}.
	\end{itemize}
\item Proyecto: \textit{Front-end} en \textit{trading exchange}.
	\begin{itemize}%
	\item Duración: Julio 2018 - actual.
	\item Descripción: Desarrollo de front-end en plataforma centralizada para trading de criptoactivos.
	\item Responsabilidades y logros: Product owner.
	\end{itemize}
\item Artículos de finanzas en el blog de CoinFabrik:
	\begin{itemize}%
	\item Responsabilidades y tareas desarrolladas: investigación y escritura.
	\item Articles:
		\begin{itemize}
		\item \href{https://blog.coinfabrik.com/what-i-have-learned-from-my-arbitrage-experiences-with-cryptoassets/}{\textit{What I have learned from my arbitrage experiences with cryptoassets}}
		\item \href{https://blog.coinfabrik.com/analyzing-blockchain-networks-with-metcafes-and-odlyzkos-laws/}{\textit{Analyzing Blockchain Networks with Metcalfe’s and Odlyzko’s laws}}
		\item \href{https://blog.coinfabrik.com/a-review-on-cryptoasset-valuation-frameworks/}{\textit{A review on cryptoasset valuation frameworks}}
		\item \href{https://blog.coinfabrik.com/what-i-have-learned-from-my-arbitrage-experiences-with-cryptoassets/}{\textit{What I have learned from my arbitrage experiences with cryptoassets}}
		\end{itemize}	
	\end{itemize}
\end{itemize}
}

%\subsection{Vocational}
\cventry{2014--actual}{Becario Profesional}{Departamento de Mecánica Computacional en Comisión Nacional de Energía Atómica}{San Carlos de Bariloche}{}
{Proyectos de ingeniería básica para reactores nucleares de investigación.\newline{}%
Desarrollo de códigos de cálculo termohidráulico.\newline{}%
Director: Dr. Enzo A. Dari (\href{mailto:darie@cab.cnea.gov.ar}{darie@cab.cnea.gov.ar}), Co-director: Dr. Mariano Cantero (\href{mailto:mcantero@cab.cnea.gov.ar}{mcantero@cab.cnea.gov.ar}).\newline{}%
Tareas desarrolladas:%
\begin{itemize}%
\item Validación de la línea de cálculo para el modelado del Segundo Sistema de Parada del reactor RA-10.
\item Análisis multiescala del Segundo Sistema de Parada del reactor RA-10.
\item Simulaciones fluidodinámicas de flujo bifásico mediante las técnicas de \textit{volume of fluid} utilizando OpenFOAM y \textit{level-set} utilizando Par-GPFEP.
\item Desarrollo del código maestro Newton para acoplamiento explícito e implícito de programas de cálculo.
\item Acoplamiento de códigos neutrónicos (PUMA, Fermi) y códigos termohidráulicos (RELAP5, Par-GPFEP y otros códigos de desarrollo propio).
% \item Implementación de sistemas \texttt{Git} de control de versiones a códigos de cálculo y documentación técnica.
\end{itemize}}


\cventry{2014}{Consultor de ingeniería}{SIC-TEC}{Mendoza}{}
{Modelado de carga de viento sobre estructuras en construcción utilizando OpenFOAM.\newline{}
Referencias: Ing. Eduardo Tano (\href{mailto:tano@sic-tec.com.ar}{tano@sic-tec.com.ar}).}

\cventry{2013-2014}{Becario de Grado}{División de Ingeniería Nuclear en INVAP S.E.}{San Carlos de Bariloche}{}
{Proyecto Integrador de la Carrera de Ingeniería Nuclear, con tema: \textit{Desarrollo Conceptual de un Reactor Rápido}.\newline{}
Director: Dr. Eduardo Villarino (\href{mailto:men@invap.com.ar}{men@invap.com.ar}).}

%\subsection{Miscellaneous}
%\cventry{year--year}{Job title}{Employer}{City}{}{Description}

\section{Experiencia en Enseñanza}

\cventry{2016}{Ayudante Auxiliar ad-honorem}{Matemática 2 (Matemática 2A y Métodos Numéricos), Instituto Balseiro}{San Carlos de Bariloche}{}
{Referencias: Dr. Javier Fernandez (\href{mailto:jfernand@cab.cnea.gov.ar}{jfernand@cab.cnea.gov.ar}), Dr. Enzo A. Dari (\href{mailto:darie@cab.cnea.gov.ar}{darie@cab.cnea.gov.ar}).}

\section{Idiomas}
\cvitemwithcomment{Inglés}{Lee y escribe con fluidez. Habla intermedio.}{}
\cvitemwithcomment{Francés}{Habilidades de comunicación básicas.}{Certificado internacional A1 en 2015.}

% \section{Lenguages de programación}
% \cvdoubleitem{\texttt{C}}{Avanzado} {\texttt{C++}}{Avanzado}
% \cvdoubleitem{\texttt{Fortran}}{Intermedio} {\texttt{Latex}}{Intermedio}
% \cvdoubleitem{\texttt{Octave}}{Avanzado}{\texttt{Python}}{Avanzado}
% \cvitem{\texttt{Scripting}}{Intermedio}


\section{Conocimientos técnicos}

\subsection{Lenguajes de programación científica}

\cvdoubleitem
{\texttt{C}}{Avanzado}
{\texttt{C++}}{Avanzado}

\cvdoubleitem
{\texttt{CUDA C}}{Intermedio}
{\texttt{Fortran}}{Intermedio}

\cvdoubleitem
{\texttt{Octave}}{Intermedio} 
{\texttt{Python}}{Avanzado}

\cvitem
{\texttt{Scripting}}{Intermedio}

\subsection{Programación back-end}
\cvdoubleitem
{\texttt{MySQL}}{Básico}
{\texttt{PHP}}{Básico}

\subsection{Programación front-end}
\cvdoubleitem
{\texttt{CSS}}{Intermedio}
{\texttt{HTML}}{Intermedio}

\cvdoubleitem
{\texttt{Javascript}}{Básico}
{\texttt{Markdown}}{Básico}

\subsection{Programación Android}
\cvdoubleitem
{\texttt{Kivy}}{Intermedio}
{\texttt{Unity 3D}}{Básico}

\subsection{Otros}

\cvlistitem{Documentación científica y técnica: {Latex, Microsoft Office}}
\cvlistitem{Librerías científicas: {cuRAND, GNU Scientific Library (GSL), Matplotlib, NumPy, OpenMP, OpenMPI, Pandas, PETSc, PyBrain, PyFoam, SLEPc, Scikit-learn, ScyPy, Thrust}}
\cvlistitem{Sistemas de control de versiones de software: {Git, Mercurial}}
\cvlistitem{Sistemas operativos: Debian GNU/Linux, Microsoft Windows}
\cvlistitem{Sotware científico: {GNU Project Debugger (GDB), Gmsh, Gnuplot, Mathematica, MATLAB, OpenFOAM, Origin, Paraview, SALOME}}




\section{Becas}

\cvitem{2017}{Beca para cursar \textit{Latin American Summer School in Computational Neuroscience LACONEU 2017 (Escuela de Verano Latinoamericana en Neurociencia computacional LACONEU 2017)}.}

\cvitem{2014--actual}{Beca A1P para perfeccionamiento profesional en el Departamento de Mecánica Computacional de la Comisión Nacional de la Energía Atómica.}

\cvitem{2011--2014}{Beca de grado para cursar la carrera de Ingeniería Nuclear en el Instituto Balseiro.}




\section{Cursos de especialización}

\subsection{Cursos tomados en maestría}

\cvitem{2016}{\textit{Modelado de sistemas termohidráulicos en reactores mediante códigos de planta}, Profesor: Dr. Pablo Zanocco, 80 hs, Instituto Balseiro, Universidad Nacional de Cuyo y Comisión Nacional de Energía Atómica, San Carlos de Bariloche.}

\cvitem{2015}{\textit{Introducción al cómputo en placas gráficas}, Profesor: Dr. Flavio D. Colavecchia, 64 hs, Instituto Balseiro, Universidad Nacional de Cuyo y Comisión Nacional de Energía Atómica, San Carlos de Bariloche.}

\cvitem{2015}{\textit{Introducción al procesamiento distribuido}, Profesor: Dr. Enzo A. Dari, 60 hs, Instituto Balseiro, Universidad Nacional de Cuyo, San Carlos de Bariloche.}

\cvitem{2015}{\textit{Redes Neuronales}, Profesor: Dr. Germán Mato, 128 hs, Instituto Balseiro, Universidad Nacional de Cuyo y Comisión Nacional de Energía Atómica, San Carlos de Bariloche.}

\cvitem{2014}{\textit{Método de elementos finitos}, Profesor: Dr. Enzo Dari, 120 hs, Instituto Balseiro, Universidad Nacional de Cuyo y Comisión Nacional de Energía Atómica, San Carlos de Bariloche.}

\cvitem{2014}{\textit{Métodos numéricos en mecánica de fluidos}, Profesor: Dr. Federico Teruel, 80 hs, Instituto Balseiro, Universidad Nacional de Cuyo y Comisión Nacional de Energía Atómica, San Carlos de Bariloche.}

\cvitem{2013}{\textit{Cálculo y análisis de reactores}, Profesor: Dr. Edmundo Lopasso, 80 hs, Instituto Balseiro, Universidad Nacional de Cuyo y Comisión Nacional de Energía Atómica, San Carlos de Bariloche.}

\subsection{Other courses}

\cvitem{2018}{\textit{Data Analysis with Python}, Curso en línea tomado en cognitiveclass.ai, una iniciativa de IBM. La autenticidad del certificado puede validarse en:
{\href{https://courses.cognitiveclass.ai/certificates/1e4b7f8f9b9c4258927b7e663f3165b5}{https://courses.cognitiveclass.ai/certificates/1e4b7f8f9b9c4258927b7e663f3165b5}}
}

\cvitem{2018}{\textit{Deep learning with tensorflow}, Curso en línea tomado en cognitiveclass.ai, una iniciativa de IBM. La autenticidad del certificado puede validarse en:
{\href{https://courses.cognitiveclass.ai/certificates/3043c010ae9745818c7917e771f79954}{https://courses.cognitiveclass.ai/certificates/3043c010ae9745818c7917e771f79954}}
}

\iffalse

\subsection{Otros cursos}

\cvitem{2007}{\textit{Filosofía}, Profesor: Dr. Pablo Cerolini, 120 hs, Facultad de Psicología, Universidad Nacional de Rosario, Rosario.}

\cvitem{2007}{\textit{Desarrollos Psicológicos Contemporáneos}, Profesor: Dra. M. Olcese, 120 hs, Facultad de Psicología, Universidad Nacional de Rosario, Rosario.}

\cvitem{2007}{\textit{Psicología}, Profesor: Dr. Antonio Gentile, 120 hs, Facultad de Psicología, Universidad Nacional de Rosario, Rosario.}

\cvitem{2007}{\textit{Lingüística}, Profesor: Lic. L. Cisneros, 120 hs, Facultad de Psicología, Universidad Nacional de Rosario, Rosario.}

\cvitem{2007}{\textit{Epistemología}, Profesor: Dr. Andrés Capelleti, 120
hs, Facultad de Psicología, Universidad Nacional de Rosario, Rosario.}

\cvitem{2007}{\textit{Trabajo de campo: área laboral}, 120 hs, Facultad de Psicología, Universidad Nacional de Rosario, Rosario.}

\fi


\section{Publicaciones}

\subsection{Informes técnicos en Comisión Nacional de Energía Atómica}

\cvitem{2015}{\textit{Anális hidrodinámico del Segundo Sistema de Parada del reactor RA-10}, Ludmila M. Rechiman, Mariano Cantero,  Enzo A. Dari, {Federico A. Caccia} y Andrés Chacoma, Informe Técnico CNEA IN-ATN40MC-04/2015, San Carlos de Bariloche, Argentina.}

\subsection{Publicación en revistas internacionales}

\cvitem{2017}{\textit{Three-dimensional hydrodynamic modeling of the Second Shutdown System of an experimental nuclear reactor (Modelo hidrodinámico tri-dimensional del Segundo Sistema de Parada de un reactor nuclear de experimentación)}, Ludmila M. Rechiman, Mariano Cantero, {Federico A. Caccia}, Andrés Chacoma y Enzo A. Dari, Nuclear Engineering and Design, vol 319, pp 163-175, doi: 10.1016/j.nucengdes.2017.04.024.}

\subsection{Presentaciones en congresos con publicación en actas}

\cvitem{2016}{\textit{Acoplamiento multiescala en cálculos Fluidodinámicos}, {Federico A. Caccia} y Enzo A. Dari, XXII Congreso sobre Métodos Numéricos y sus Aplicaciones ENIEF 2016, Universidad Tecnológica Nacional, Córdoba. Publicado en Mecánica Computacional Vol XXXIV, págs. 1955-1972.}

\cvitem{2016}{\textit{Validation of a multiscale model of the second shutdown system of an experimental nuclear reactor (Validación de un modelo multiescala del Segundo Sistema de Parada de un reactor experimental)}, Ludmila M. Rechiman, Mariano Cantero, {Federico A. Caccia} y Enzo A. Dari, XXII Congreso sobre Métodos Numéricos y sus Aplicaciones ENIEF 2016, Universidad Tecnológica Nacional, Córdoba. Publicado en Mecánica Computacional Vol XXXIV, págs. 2199-2215.}

\section{Congresos y cursos atendidos}

\cvitem{2017}{\textit{Evolution of neural computation (Evolución de la computación neuronal)}, Instituto Balseiro, Universidad Nacional de Cuyo, San Carlos de Bariloche.}

\cvitem{2017}{\textit{Latin American Summer School in Computational Neuroscience LACONEU 2017 (Escuela de Verano Latinoamericana en Neurociencia computacional LACONEU 2017)}, Título del proyecto: \textit{Adaptación sensorial sin plasticidad en la corteza visual V1}, Instituto de Sistemas Complejos de Valparaíso, Valparaíso, Chile.}

\cvitem{2017}{\textit{Computational Neuroscience: new trends and challenges for the 2030 (Neurociencia Computacional: Nuevas Tendencias y Desafíos para el 2030)}, Instituto de Sistemas Complejos de Valparaíso, Valparaíso, Chile.}

\cvitem{2016}{\textit{Machine Learning (Aprendizaje de máquina)}, Instituto Balseiro, Universidad Nacional de Cuyo, San Carlos de Bariloche.}

\cvitem{2016}{\textit{XXII Congreso sobre Métodos Numéricos y sus Aplicaciones ENIEF 2016}, Universidad Tecnológica Nacional, Córdoba.}

\cvitem{2015}{\textit{Plasma processing of radioactive wastes: process engineering, flue gas and solid wastes (Procesamiento por plasma de desechos radiactivos: ingeniería de procesos, gases de combustión y desechos sólidos)}, organizado por el Departamento de Materiales Nucleares, el Programa Nacional de Residuos Radiactivos y la Organización Nacional de Energía Atómica, Centro Atómico Bariloche, San Carlos de Bariloche.}

\cvitem{2014}{\textit{XXI Congreso sobre Métodos Numéricos y sus Aplicaciones ENIEF 2014}, Centro Atómico Bariloche, San Carlos de Bariloche.}

\section{Desarrollo de Software}

\cvitem{\texttt{Online trending detection}}{Online trending detection es un código de procesamiento de lenguage natural que realiza agrupamiento incremental de noticias, tweets y otros artículos de redes sociales.}

\cvitem{\texttt{Newton}}{Newton es un código maestro que resuelve acoplamiento explícitos e implícitos en cálculos no lineales, por ejemplo en acoplamientos fluidodinámicos, termohidráulicos, neutrónicos, etc (\httplink{www.github.com/fedecaccia/newton}).}

\cvitem{\texttt{Par-GPFEP}}{Par-GPFEP es un programa de elementos finitos de propósito general diseñado para resolver problemas mecánicos que involucran flujos multifásicos, modelos turbulentos, seguimiento de superficies libres, transferencia de calor, interacción fluido-estructura y otros.}


\vspace{\fill}
\hspace{0.7\linewidth}
\begin{minipage}{0.3\linewidth}
	\begin{center}
		Federico Agustín Caccia\\
		%\today
		16 de Julio de 2018
		\bigskip
	\end{center}
\end{minipage}

%\section{Extra 1}
%\cvlistitem{Item 1}
%\cvlistitem{Item 2}
%\cvlistitem{Item 3. This item is particularly long and therefore normally spans over several lines. Did you notice the indentation when the line wraps?}

%\section{Extra 2}
%\cvlistdoubleitem{Item 1}{Item 4}
%\cvlistdoubleitem{Item 2}{Item 5\cite{book1}}
%\cvlistdoubleitem{Item 3}{Item 6. Like item 3 in the single column list before, this item is particularly long to wrap over several lines.}

%\section{References}
%\begin{cvcolumns}
 % \cvcolumn{Category 1}{\begin{itemize}\item Person 1\item Person 2\item Person 3\end{itemize}}
  %\cvcolumn{Category 2}{Amongst others:\begin{itemize}\item Person 1, and\item Person 2\end{itemize}(more upon request)}
  %\cvcolumn[0.5]{All the rest \& some more}{\textit{That} person, and \textbf{those} also (all available upon request).}
%\end{cvcolumns}

% Publications from a BibTeX file without multibib
%  for numerical labels: \renewcommand{\bibliographyitemlabel}{\@biblabel{\arabic{enumiv}}}% CONSIDER MERGING WITH PREAMBLE PART
%  to redefine the heading string ("Publications"): \renewcommand{\refname}{Articles}

%%%%\nocite{*}
%%%%\bibliographystyle{plain}
%%%%\bibliography{publications} % 'publications' is the name of a BibTeX file

% Publications from a BibTeX file using the multibib package
%\section{Publications}
%\nocitebook{book1,book2}
%\bibliographystylebook{plain}
%\bibliographybook{publications}                   % 'publications' is the name of a BibTeX file
%\nocitemisc{misc1,misc2,misc3}
%\bibliographystylemisc{plain}
%\bibliographymisc{publications}                   % 'publications' is the name of a BibTeX file


% COVER LETTER

\iffalse


\clearpage
%-----       letter       ---------------------------------------------------------
% recipient data
\recipient{Company Recruitment team}{Company, Inc.\\123 somestreet\\some city}
\date{January 01, 1984}
\opening{Dear Sir or Madam,}
\closing{Yours faithfully,}
\enclosure[Attached]{curriculum vit\ae{}}          % use an optional argument to use a string other than "Enclosure", or redefine \enclname
\makelettertitle

Lorem ipsum dolor sit amet, consectetur adipiscing elit. Duis ullamcorper neque sit amet lectus facilisis sed luctus nisl iaculis. Vivamus at neque arcu, sed tempor quam. Curabitur pharetra tincidunt tincidunt. Morbi volutpat feugiat mauris, quis tempor neque vehicula volutpat. Duis tristique justo vel massa fermentum accumsan. Mauris ante elit, feugiat vestibulum tempor eget, eleifend ac ipsum. Donec scelerisque lobortis ipsum eu vestibulum. Pellentesque vel massa at felis accumsan rhoncus.

Suspendisse commodo, massa eu congue tincidunt, elit mauris pellentesque orci, cursus tempor odio nisl euismod augue. Aliquam adipiscing nibh ut odio sodales et pulvinar tortor laoreet. Mauris a accumsan ligula. Class aptent taciti sociosqu ad litora torquent per conubia nostra, per inceptos himenaeos. Suspendisse vulputate sem vehicula ipsum varius nec tempus dui dapibus. Phasellus et est urna, ut auctor erat. Sed tincidunt odio id odio aliquam mattis. Donec sapien nulla, feugiat eget adipiscing sit amet, lacinia ut dolor. Phasellus tincidunt, leo a fringilla consectetur, felis diam aliquam urna, vitae aliquet lectus orci nec velit. Vivamus dapibus varius blandit.

Duis sit amet magna ante, at sodales diam. Aenean consectetur porta risus et sagittis. Ut interdum, enim varius pellentesque tincidunt, magna libero sodales tortor, ut fermentum nunc metus a ante. Vivamus odio leo, tincidunt eu luctus ut, sollicitudin sit amet metus. Nunc sed orci lectus. Ut sodales magna sed velit volutpat sit amet pulvinar diam venenatis.

Albert Einstein discovered that $e=mc^2$ in 1905.

\[ e=\lim_{n \to \infty} \left(1+\frac{1}{n}\right)^n \]

\makeletterclosing

%\clearpage\end{CJK*}                              % if you are typesetting your resume in Chinese using CJK; the \clearpage is required for fancyhdr to work correctly with CJK, though it kills the page numbering by making \lastpage undefined

\fi


\end{document}


%% end of file `template.tex'.
